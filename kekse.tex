\section{Kekse}

\begin{recipe}{Ghanataler}
  \ingredient{250\,g Mehl}
  \ingredient[Kakao]{2\,EL Kakaopulver}
  \ingredient{1\,TL Backpulver}
  mischen, in eine Rührschüssel sieben.

  \ingredient{250\,g Zucker}
  \ingredient{1 Pkg. Vanillezucker}
  \ingredient[Zimt]{1\,TL gemahlener Zimt}
  \ingredient{1 Ei}
  \ingredient{200\,g Butter}
  hinzufügen, mit Handrührgerät mit Rührbesen zunächst auf niedriger,
  dann auf höchster Stufe gut durcharbeiten.

  Anschließend auf der bemehlten Arbeitsfläche zu einem glatten Teig
  verarbeiten und
  \ingredient[Mandel]{150\,g gemahlene Mandeln}
  unterarbeiten.

  Den Teig zu Rollen (ca. 3\,cm \o) formen und in Frischhaltefolie gewickelt
  einige Stunden oder über Nacht kalt stellen.

  Teigrollen in etwa 5\,mm dicke Scheiben schneiden und auf Backbleche
  (mit Backpapier belegt) legen.

  Die Teigtaler mit
  \ingredient{2--3\,EL Milch}
  bestreichen und mit
  \ingredient{ca. 100\,g Hagelzucker}
  bestreuen.

  Bei 180° (Umluft) bzw. 200° (Ober-"/Unterhitze) ca. 12 Minuten backen.
\end{recipe}



\begin{recipe}{Haferplätzchen}
  \ingredient{50\,g Butter}
  schmelzen, mit
  \ingredient{100\,g Zucker}
  \ingredient[Zimt]{1\,TL Zimt}
  cremig rühren.

  \ingredient[Orange]{2\,EL Orangensaft}
  \ingredient[Rum]{1\,TL Rum}
  unterrühren.

  \ingredient{50\,g Mehl}
  \ingredient{25\,g Haferflocken}
  \ingredient[Mandel]{50\,g gehobelte Mandeln}
  mischen, unterheben.

  1 Stunde abgedeckt kalt stellen.

  Kleine Häufchen aufs Backblech setzen.

  Bei 175° (Ober-"/Unterhitze) ca. 12 Minuten backen.

  Kekse abkühlen lassen.

  \ingredient[Schokolade]{100\,g Zartbitterkuvertüre}
  schmelzen, Kekse zur Hälfte eintauchen.
\end{recipe}



\begin{recipe}{Ingwergebäck}
  \ingredient{125\,g Margarine}
  schaumig rühren. Nach und nach
  \ingredient{200\,g Zucker}
  \ingredient{1 Pkg. Vanillezucker}
  \ingredient[Ingwer]{2\,TL gemahlener Ingwer}
  \ingredient{4 Eier}
  hinzugeben.
  \ingredient{250\,g Mehl}
  \ingredient{3\,g Backpulver}
  mischen, sieben. Mit
  \ingredient[Schokolade]{250\,g Raspelschokolade}
  esslöffelweise unterrühren.
  \ingredient[Rosine]{200\,g Rosinen}
  unterheben.

  Den Teig auf ein gefettetes Backbleck streichen.

  Bei 175°--200° (Ober-"/Unterhitze) ca. 20--25 Minuten backen.

  Das erkaltete Gebäck in Stücke schneiden. Mit
  \ingredient{150\,g Kuvertüre halbbitter}
  bestreichen.
\end{recipe}



\begin{recipe}{Kardamommonde}
  \ingredient{150\,g Butter}
  \ingredient{250\,g Zucker}
  \ingredient{1 Pkg. Vanillezucker}
  \ingredient{2 Eier}
  schaumig rühren.
  \ingredient[Kardamom]{1\,TL gemahlener Kardamom}
  \ingredient[Zimt]{1\,Msp. gemahlener Zimt}
  \ingredient[Mandel]{220\,g gemahlene Mandeln}
  \ingredient{250\,g Mehl}
  unterkneten, 1--2 Stunden kühl stellen.

  Teig auf wenig Mehl ca. 3\,mm dick ausrollen. Monde ausrollen, auf ein
  gefettetes Blech legen.

  \ingredient{1 Ei}
  verquirlen, Monde damit bestreichen.
  \ingredient[Mandel]{30\,g gemahlene Mandeln}
  über die Monde streuen.

  Bei 175° (Ober-"/Unterhitze) ca. 10--15 Minuten backen.
\end{recipe}



\begin{recipe}{Kokostaler}
  \ingredient{250\,g Mehl}
  \ingredient{\sfrac{1}{2}\,TL Backpulver}
  mischen und in eine Rührschüssel sieben.

  \ingredient{250\,g Zucker}
  \ingredient{1 Pkg. Vanillezucker}
  \ingredient{5\,Trpf. Bittermandel-Aroma}
  \ingredient{1 Ei}
  \ingredient{250\,g weiche Butter}
  hinzufügen, mit Handrührgerät mit Rührbesen zunächst auf niedriger,
  dann auf höchster Stufe gut durcharbeiten.

  \ingredient[Kokos]{200\,g Kokosraspel}
  unterarbeiten.
  
  Anschließend auf der bemehlten Arbeitsfläche zu einem glatten Teig
  verkneten.
  
  Den Teig zu Rollen (ca. 4\,cm \o) formen und in Frischhaltefolie gewickelt
  einige Stunden oder über Nacht kalt stellen.
  
  Teigrollen in etwa 5\,mm dicke Scheiben schneiden und auf Backbleche
  (mit Backpapier belegt) legen.

  Bei 160--180° (Umluft) bzw. 180--200° (Ober-"/Unterhitze) ca. 12 Minuten backen.
\end{recipe}



\begin{recipe}{Orangenlikörplätzchen}
  \ingredient{150\,g Zucker}
  \ingredient{1 Pkg. Vanillezucker}
  \ingredient{1 Prise Salz}
  \ingredient{200\,g Butter}
  cremig rühren.

  \ingredient{2 Eier}
  \ingredient[Orange]{1 Orange (Schale)}
  \ingredient[Orangenlikör]{1\,EL Orangenlikör}
  zufügen.

  \ingredient{375\,g Mehl}
  \ingredient[Mandel]{125\,g gemahlene Mandeln}
  mischen und unterkneten.

  Aus dem Teig ca. 4\,cm dicke Rollen formen und in Folie gewickelt 2--3
  Stunden kühl stellen.

  Die Rollen in 1,5\,cm dicke Scheiben schneiden und diese auf ein mit
  Backpapier belegtes Blech legen.

  Bei 160° (Umluft) bzw. 180° (Ober-"/Unterhitze) ca. 12--15 Minuten backen.

  \ingredient[Orangenlikör]{4--6\,EL Orangenlikör}
  auf die noch warmen Plätzchen streichen.
  \ingredient{Puderzucker}
  dick über die Plätzchen stäuben.
\end{recipe}



\begin{recipe}{Shortbread-Imitat-Scheiben}
  \ingredient{300\,g Weizenmehl}
  in eine Rührschüssel sieben.
  
  \ingredient{30\,g Hartweizengries}
  \ingredient{120\,g Zucker}
  \ingredient{1 Pkg. Vanillezucker}
  \ingredient{\sfrac{1}{2}\,TL Salz}
  \ingredient{2 Eigelb}
  \ingredient{200\,g Butter}
  \ingredient{2\,EL kaltes Wasser}
  hinzufügen, mit Handrührgerät mit Knethaken auf höchster Stufe gut
  durcharbeiten.

  Anschließend auf der bemehlten Arbeitsfläche zu einem glatten Teig
  verkneten.
  
  Den Teig in zwei Rollen (ca. 25\,cm Länge) formen und in
  Frischhaltefolie gewickelt mindestens eine Stunde kalt stellen.

  Die Teigrollen mit Wasser bestreichen und in
  \ingredient{2\,EL Zucker}
  wälzen, andrücken.
  
  Teigrollen in etwa 1\,cm dicke Scheiben schneiden, dabei die Rolllen
  drehen, um gleichmäßige Scheiben zu erhalten.

  Die Teigscheiben auf Backbleche (mit Backpapier belegt) legen. Nach
  Belieben mit einer Gabel Muster in die Teigscheiben stechen.

  Bei 160° (Umluft) bzw. 180° (Ober-"/Unterhitze) ca. 10--15 Minuten backen.
\end{recipe}



\begin{recipe}{Straciatella-Wölkchen}
  \ingredient[Eiweiß]{5 Eiweiß}
  steif schlagen.

  \ingredient{200\,g Puderzucker}
  \ingredient{3\, Pkg. Vanillezucker}
  \ingredient[Kokos]{200\,g Kokosflocken}
  \ingredient[Mandel]{200\,g gemahlene Mandeln}
  \ingredient[Schokolade]{75\,g Raspelschokolade}
  mischen, unterheben.

  \modification{\index[ingredient]{Zimt} Nach Geschmack Zimt hinzugeben.}
  
  Kleine Kugeln formen und aufs Backpapier setzen.

  Bei 160° (Ober-"/Unterhitze) je nach Größe der Kugeln 15--30 Minuten
  backen bis sie leicht braun werden.
\end{recipe}



\begin{recipe}{Vanillekipferl}
  \ingredient{250\,g Mehl}
  \ingredient{1\,Msp. Backpulver}
  mischen, in eine Rührschüssel sieben.
  
  \ingredient{125\,g Zucker}
  \ingredient{1 Pkg. Vanillezucker}
  \ingredient{3 Eigelb}
  \ingredient{200\,g Butter}
  \ingredient[Mandel]{125\,g gemahlene Mandeln}
  hinzufügen, mit Handrührgerät mit Knethaken zunächst kurz auf
  niedrigster, dann auf höchster Stufe gut durcharbeiten.
  Anschließend auf der leicht bemehlten Arbeitsfläche zu einem glatten
  Teig verkneten. Den Teig in Folie gewickelt eine Zeit lang kalt
  stellen.

  Aus dem Teig auf der leicht bemehlten Arbeitsfläche etwa daumendicke
  Rollen formen, gut 2\,cm lange Stücke davon abschneiden und diese zu
  etwa 5\,cm langen Roclen formen. Dabei die Enden etwss dünner rollen.
  Die Teigrollen leicht gebogen (hörnchenförmig) auf Backbleche (mit
  Backpapier belegt) legen.

  Bei 160°--180° (Umluft) bzw. 180°--200° (Ober-"/Unterhitze) ca. 10 Minuten backen.

  Die Kipferl mit dem Backpapier von den Backblechen auf einen
  Kuchenrost ziehen.
  
  \ingredient{50\,g Puderzucker}
  \ingredient{1 Pkg. Vanillezucker}
  mischen, sieben. Die heissen Kipferl darin wenden und auf einem
  Kuchenrost erkalten lassen.
\end{recipe}



\begin{recipe}{Walnusshäufchen}
  \ingredient[Eiweiß]{2 Eiweiß}
  steif schlagen.

  \ingredient{100\,g Puderzucker}
  \ingredient{1\,TL Sahnesteif}
  \ingredient[Walnuss]{200\,g gehackte Walnüsse}
  unterheben.

  Kleine Häufchen aufs Backpapier setzen.

  Bei 140° (Ober-"/Unterhitze) 20 Minuten backen.
\end{recipe}



\begin{recipe}{Walnussplätzchen}
  \ingredient{300\,g Mehl}
  in eine Rührschüssel sieben.

  \ingredient{200\,g Zucker}
  \ingredient{1 Pkg. Vanillezucker}
  \ingredient{1 Fl. Rum-Aroma}
  \ingredient{3 Trpf. Bittermandel-Aroma}
  \ingredient[Kardamom]{1 Msp. gemahlener Kardamom}
  \ingredient{1 Prise Salz}
  \ingredient{1 Ei}
  \ingredient{200\,g Butter}
  hinzufügen, mit Handrührgerät mit Rührbesen zunächst auf niedriger,
  dann auf höchster Stufe gut durcharbeiten,
  \modification{Anstelle von Kardamom kann auch
    \index[ingredient]{Zimt} Zimt verwendet werden.}

  \ingredient[Walnuss]{150\,g gemahlene Walnusskerne}
  unterarbeiten.

  Anschließend auf der bemehlten Arbeitsfläche zu einem glatten Teig
  verkneten und in Frischhaltefolie gewickelt einige Stunden oder über
  Nacht kalt stellen.

  Den Teig portionsweise auf der bemehlten Arbeitsfläche dünn ausrollen
  und mit einer runden Ausstechform (ca. 4\,cm \o) etwa 100 Plätzchen
  ausstechen. Teigplätzchen auf Backbleche (mit Backpapier belegt)
  legen.

  Bei 160° (Umluft) bzw. 180° (Ober-"/Unterhitze) ca. 10 Minuten backen.

  Plätzchen erkalten lassen.

  \ingredient[Johannisbeere]{2--3\,EL rotes Johannisbeergelee}
  glatt rühren.
  Die Hälfte der Plätzchen auf der Unterseite mit dem Gelee bestreichen,
  die restlichen Plätzchen darauf legen und gut andrücken.

  \ingredient[Schokolade]{50\,g Halbbitter-Kuvertüre}
  im Wasserbad schmelzen.
  \ingredient[Walnuss]{200\,g halbierte Walnusskerne}
  jeweils zur Hälfte in die Kuvertüre tauchen und auf Backpapier legen.
  Kuvertüre fest werden lassen.

  \ingredient{150--200\,g Zitronenglasur}
  nach Packungsanleitung auflösen, die Plätzchen damit bestreichen und
  mit den Walnusskernhälften garnieren.
\end{recipe}



\begin{recipe}{Zitronetten}
  \ingredient{375\,g Weizenmehl}
  \ingredient{1\,TL Backpulver}
  mischen, in eine Rührschüssel sieben.

  \ingredient{125\,g Zucker}
  \ingredient{1 Pkg. Vanillezucker}
  \ingredient[Zimt]{1\,TL gemahlener Zimt}
  \ingredient{1 Ei}
  \ingredient{1\,EL Milch}
  \ingredient[Mandel]{100\,g gemahlene Mandeln}
  \ingredient{200\,g Butter}
  hinzufügen, mit Handrührgerät mit Knethaken zunächst auf niedriger,
  dann auf höchster Stufe gut durcharbeiten, dabei evtl.
  \ingredient{1--2\,EL Milch}
  hinzufügen.

  Anschließend auf der bemehlten Arbeitsfläche zu einem glatten Teig
  verarbeiten.

  Den Teig zu Rollen (ca. 3\,cm \o) formen und in Frischhaltefolie gewickelt
  einige Stunden oder über Nacht kalt stellen.

  Teigrollen in etwa 5\,mm dicke Scheiben schneiden und auf Backbleche
  (mit Backpapier belegt) legen.

  Bei 180° (Umluft) bzw. 200° (Ober-"/Unterhitze) ca. 12--15 Minuten backen.

  Plätzchen erkalten lassen.

  \ingredient[Aprikose]{2\,EL Aprikosenkonfitüre} mit
  \ingredient[Zitrone]{1\,EL~Zitronensaft oder
    3\,EL~Zitronengelee}
  verrühren.
  Die Hälfte der Plätzchen auf der Unterseite mit der Füllung bestreichen,
  die restlichen Plätzchen darauf legen und gut andrücken.

  \ingredient{200\,g gesiebter Puderzucker}
  \ingredient[Zitrone]{ca. 5\,EL Zitronensaft}
  zu einer dickflüssigen Masse verrühren.
  Die Plätzchen damit bestreichen und mit
  \ingredient[Pistazie]{60\,g Pistazienkerne, in Stifte geschnitten}
  bestreuen. Guss trocknen lassen.
\end{recipe}
