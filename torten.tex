\section{Torten}

\begin{recipe}{Ananas-Sahnetorte}
  \ingredient[Ananas]{2 Dosen Ananas (Scheiben)}
  als Garnitur 16 Stücke schneiden und zur Seite stellen.

  \ingredient{Gelatine (weiß)}
  Die restlichen Ananas"=Ringe klein schneiden und mit dem Saft abmessen.
  Entsprechende Menge Gelatine nach Vorschrift zubereiten und unter die
  Ananas rühren.

  \ingredient{100\,g Konditorsahne}
  schlagen und unter die Ananasmasse heben, sobald diese beginnt steif
  zu werden.

  \ingredient{1 Wiener Boden (Edeka)}
  in drei Böden schneiden. Tortenring auf eine Platte stellen, den
  ersten Boden einlegen und die Hälfte der Ananas"=Sahnemasse einfüllen.
  Den zweiten Boden auflegen und die restliche Ananas"=Sahnemasse darauf
  verteilen. Den letzten Boden auflegen und einige Stunden kalt stellen.

  \ingredient{200\,g Konditorsahne}
  \ingredient{1 Pkg. Sahnesteif}
  schlagen. Tortenring entfernen und die Torte mit der Sahne und den
  Ananasstücken verzieren.
\end{recipe}



\begin{recipe}{Nußtorte}
  \ingredient[Haselnuss]{300\,g gemahlene Haselnüsse}
  \ingredient{225\,g Zucker}
  \ingredient{5 Eier}
  Zucker und Eigelb verrühren, Eiweiß schlagen. Eiweiß und Haselnüsse
  abwechselnd unter die Eigelbmasse rühren.

  \ingredient{1\sfrac{1}{2} TL Backpulver}
  \ingredient{1 EL Mehl}
  mischen und unterheben.

  Den Teig in eine Springform füllen (Boden mit Papier auslegen) und bei
  170° (Ober-"/Unterhitze) ca. 60 Minuten backen. Falls der Boden zu
  dunkel wird, mit Papier abdecken. Den Boden abkühlen lassen und
  halbieren.

  \ingredient{300\,g Konditorsahne}
  \ingredient{1 Pkg. Sahnesteif}
  schlagen. Boden mit \sfrac{2}{3} der Sahne füllen und mit dem Rest
  einstreichen.

  \ingredient[Haselnuss]{16 Haselnüsse}
  zum garnieren.  
\end{recipe}



\begin{recipe}{Ostfriesentorte mit Rumrosinen}
  \ingredient{800\,ml Sahne}
  \ingredient{2 EL Puderzucker}
  steif schlagen und in vier Portionen aufteilen.

  Von
  \ingredient[Rumrosinen]{2 Tassen in Rum eingelegte Rosinen}
  einige zum Verzieren zur Seite legen.

  \ingredient{1 Biskuitboden, dunkel}
  zweimal durchschneiden.

  Den unteren Boden mit einem Tortenrand umlegen und mit einer Tasse
  Rumrosinen bestreuen. \sfrac{1}{4} der Sahne darüber streichen.

  Darauf dann den zweiten Boden legen, ebenfalls mit einer
  Tasse Rumrosinen belegen und mit \sfrac{1}{4} Sahne bestreichen.

  Den dritten Boden auf den zweiten setzen und auch darauf \sfrac{1}{4}
  Sahne verteilen.

  Mit der restlichen Sahne und den Rumrosinen die Torte nach Belieben verzieren. 
\end{recipe}


\begin{recipe}{Pariser Trüffeltorte}
  \ingredient{990\,g Konditorsahne}
  aufkochen.
  
  \ingredient[Schokolade]{300\,g Bitterschokolade (mind. 75\%)}
  in der heißen Sahne auflösen, über Nacht in den Kühl"-schrank stellen.

  \ingredient{1 Pkg. Sahnesteif}
  mit der Schokosahne aufschlagen.

  \ingredient{1 dunkler Wiener Boden (Edeka)}
  in drei Böden schneiden. Tortenring auf eine Platte stellen, den
  ersten Boden einlegen und ein Drittel der Sahnemasse einfüllen. Den
  zweiten Boden auflegen und das zweite Drittel der Sahnemasse darauf
  verteilen. Den letzten Boden auflegen, Tortenring entfernen und mit
  der restlichen Sahnemasse und
  \ingredient[Schokotrüffel]{16~Minitrüffel oder 16~Schokoblättchen}
  verzieren.
\end{recipe}
