\section{Kuchen}

\begin{recipe}{Ananas-Marzipankuchen}
  \ingredient[Marzipan]{200\,g Marzipan-Rohmasse}
  \ingredient{175\,g Margarine}
  zu einer geschmeidigen Masse verrühren.
  
  \ingredient{175\,g Zucker}
  \ingredient{1 Pkg. Vanillin-Zucker}
  \ingredient{3 Eier}
  nach und nach hinzugeben.
  
  \ingredient{300\,g Mehl}
  \ingredient{2 gestr. TL Backpulver}
  mischen, sieben, esslöffelweise unterrühren.
  
  \ingredient[Ananas]{200\,g Ananas}
  unterrühren.

  Den Teig in eine mit Backpapier ausgelegte Kastenform füllen.

  Bei 175°--200° (Ober-"/Unterhitze) ca. 60-70 Minuten im
  vorgeheizten Backofen backen.

  Nach dem Erkalten mit
  \ingredient[Schokolade]{Schokoladenguss}
  überziehen, nach Wunsch mit Ananasstücken garnieren.
\end{recipe}



\begin{recipe}{Apfelkuchen, sehr fein}
  \ingredient{250\,g weiche Butter}
  mit Handrührgerät geschmeidig rühren.

  \ingredient{250\,g Zucker}
  \ingredient{1 Pkg. Vanillin-Zucker}
  \ingredient{1 Prise Salz}
  nach und nach hinzufügen, rühren bis eine gebundene Masse entstanden
  ist.

  \ingredient{8 Tropfen Bittermandelaroma}
  \ingredient{6 Eier (M)}
  nach und nach unterrühren (30 Sekunden pro Ei).

  \ingredient{400\,g Weizenmehl}
  \ingredient{4 TL Backpulver}
  mischen, sieben, abwechselnd mit
  \ingredient{2--4 EL Milch}
  kurz auf mittlerer Stufe unterrühren (nur so viel Milch verwenden,
  dass der Teig schwer reißend vom Löffel fällt).

  \sfrac{3}{4} des Teiges auf ein ein gefettetes Backblech geben und
  glatt streichen.

  \ingredient[Apfel]{1,5\,kg Äpfel}
  schälen, entkernen, in Scheiben schneiden, auf den Teig legen.

  Restlichen Teig mit
  \ingredient{6--8 EL Milch}
  verdünnen und über die Äpfel geben.

  \ingredient[Mandel]{150\, g gehobelte Mandeln}
  auf dem Teig verteilen, dünn mit
  \ingredient{Zucker}
  bestreuen,
  \ingredient{Butter}
  in Flöckchen darauf setzen.

  Bei 160° (Umluft) bzw. 180° (Ober-"/Unterhitze) ca. 40-50 Minuten im
  vorgeheizten Backofen backen.  
  
\end{recipe}



\begin{recipe}{Dresdner Eierschecke}
  \ingredient{80\,ml Milch (handwarm)}
  \ingredient{100\,g Weizenmehl}
  \ingredient{15\,g Hefe}
  zu einem weichen Teig verrühren. 30 Minuten ruhen lassen.

  \ingredient{100\,g Weizenmehl}
  \ingredient{40\,g Butter}
  \ingredient{40\,g Zucker}
  \ingredient{2 Eier}
  \ingredient{1 Prise Salz}
  hinzugeben, kräftig durchkneten. Teig abdecken und weitere 30 Minuten
  ruhen lassen.

  Während der Ruhezeit
  \ingredient{60\,g Butter}
  flüssig werden lassen und mit
  \ingredient[Quark]{400\,g Speisequark}
  \ingredient{60\,g Zucker}
  \ingredient{20\,g Weizenmehl}
  \ingredient{20\,g Puddingpulver Vanille}
  \ingredient{40\,ml Milch}
  \ingredient{2 Eier}
  zu einer glatten Masse rühren.
  \ingredient{1 Prise Salz}
  \ingredient[Zitrone]{1 Spritzer Zitronensaft}
  dazugeben.

  Hefeteig ausrollen und auf gefettetes Backblech (30cm $\times$ 40cm, mit hohem
  Rand) geben. Am Rand den Teig gut andrücken und den Boden mehrfach mit
  einer Gabel einstechen. Die Quarkmasse auf den Teig geben und
  gleichmäßig verteilen.
  
  \ingredient{370\,ml Milch}
  \ingredient{130\,g Zucker}
  zum Kochen bringen.

  \ingredient{100\,ml Milch}
  \ingredient{70\,g Puddingpulver Vanille}
  anrühren, zur gezuckerten Milch geben, einmal aufkochen lassen, dann
  vom Herd nehmen.
  
  \ingredient{7 Eigelb}
  \ingredient{230\,g Butter}
  in den heißen Pudding geben, gut verrühren.
  
  \ingredient{7 Eiweiß}
  aufschlagen, dabei nach und nach
  \ingredient{110\,g Zucker}
  \ingredient{1 Prise Salz}
  zugeben. Unter den steifen Eischnee den Pudding unterheben. Die Masse
  auf den Quark geben und glatt streichen.

  Im vorgeheizten Ofen bei 190° (Ober-"/Unterhitze) für 20 Minuten
  backen, danach auf 160° reduzieren und weitere 20 Minuten backen
  (Gesamtbackzeit mindestens 40 Minuten). Die Eierschecke ist fertig,
  sobald die Oberfläche der Masse elastisch ist. Auskühlen lassen und
  mit etwas Zucker bestreuen.
  
\end{recipe}



\begin{recipe}{Feines Früchtebrot}
  \ingredient[Feige]{125\,g getrocknete Feigen}
  \ingredient[Dattel]{100\,g getrocknete Datteln}
  in feine Streifen schneiden.
  
  \ingredient[Sultanine]{125\,g Sultaninen}
  \ingredient[Orangeat]{100\,g Orangeat}
  \ingredient[Zitronat]{100\,g Zitronat}
  \ingredient[Haselnuss]{125\,g Haselnüsse}
  \ingredient[Walnuss]{7 große Walnüsse}
  grob bis mittelfein hacken.
  
  \ingredient{3 Eier}
  \ingredient{125\,g Zucker}
  schaumig rühren, Früchte und Nüsse zugeben.
  
  \ingredient[Rum]{2\,EL Rum}
  unterrühren.
  
  \ingredient{125\,g Mehl}
  \ingredient[Zimt]{1\,TL gemahlener Zimt}
  \ingredient{1\,TL Backpulver}
  mischen, über den Teig sieben, verkneten.

  Den Teig in eine gefettete (und optional mit Semmelbröseln
  ausgestreute) Kastenform von ca. 30\,cm Länge füllen.
  
  Im vorgeheizten Ofen bei 150° (Ober-"/Unterhitze) für 60 -- 90 Minuten
  backen. Nach einer Stunde Backzeit die Oberfläche evtl. mit Alufolie
  abdecken, damit sie nicht zu dunkel wird.

  Danach mindestens 15 Minuten ruhen lassen, dann aus der Form lösen und
  auf einem Kuchengitter erkalten lassen.
\end{recipe}


\begin{recipe}{Mandarinenblechkuchen}
  \ingredient{350\,g Mehl}
  \ingredient{200\,g Butter}
  \ingredient{175\,g Zucker}
  \ingredient{2 Eier}
  \ingredient{1 Pkg. Vanillezucker}
  \ingredient{\sfrac{1}{2} Pkg. Backpulver}
  zu einem Teig verrühren, auf dem Blech verstreichen.
  
  \ingredient{500\,ml Milch}
  \ingredient{1 Pkg. Puddingpulver Vanille-Geschmack}
  \ingredient{2\,EL Zucker}
  Pudding kochen, abkühlen lassen.
  
  \ingredient[Schmand]{500\,g Schmand}
  \ingredient[Mandarine]{700\,g Mandarinen}
  unter den Pudding ziehen, auf dem Teig verteilen.
  
  \ingredient{200\,g Butter}
  \ingredient{100\,g Mehl}
  \ingredient[Kokos]{150\,g Kokosraspeln}
  \ingredient{175\,g Zucker}
  zu Teig verrühren, als Streussel auf der Füllung verteilen.
  
  Bei 170° (Umluft) ca. 25-35 Minuten backen.  
\end{recipe}
