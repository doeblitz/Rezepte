\section{Muffins}

\begin{recipe}{Muffins (Basisrezept)}
  \ingredient{200\,g Weizenmehl}
  \ingredient{2 TL Backpulver}
  vermischen, in eine Rührschüssel sieben.
  
  \ingredient{100\,g Zucker}
  \ingredient{1 Pkg. Vanillezucker}
  \ingredient{125\,g Butter, weich}
  \ingredient{3 EL Milch}
  \ingredient{3 Eier}
  dazugeben und zu einer glatten Masse verrühren.

  Bei 180° (Ober-"/Unterhitze) ca. 15 Minuten im vorgeheizten
  Backofen backen.
  
  Ergibt 12 Stück.
\end{recipe}
  
\begin{recipe}{American Chocolate-Chip-Muffins}
  \ingredient{300\,g Weizenmehl}
  \ingredient{2\sfrac{1}{4} TL Backpulver}
  \ingredient{4 EL Kakaopulver}
  \ingredient{115\,g Brauner Zucker}
  \ingredient{265\,g gehackte Bitterschokolade}
  vermischen, in die Mitte eine Mulde drücken.

  \ingredient{2 Eier}
  \ingredient{375 ml Buttermilch}
  in einer anderen Schüssel verquirlen, mit
  \ingredient{90\,g Butter, zerlassen}
  in die Mulde gießen und mit einem Metallöffel verrühren, bis gerade
  ein Teig entsteht. Der Teig sollte noch Klümpchen haben.
  
  Die Formen zu \sfrac{3}{4} mit Teig füllen,
  \ingredient{2 EL geh. Bitterschokolade}
  drüberstreuen.

  Bei 210° (Ober-"/Unterhitze) ca. 30 Minuten im vorgeheizten
  Backofen backen. Gartest machen. 5 Minuten in der Form
  abkühlen lassen, dann herauslösen und auf einem Gitter ganz
  auskühlen lassen.

  Vor dem Servieren mit Puderzucker bestreuen.
  
  Ergibt 12 Stück.
\end{recipe}

\begin{recipe}{Eierlikör-Schoko-Muffins}
  \ingredient{2 Eier}
  schaumig schlagen.

  Mit
  \ingredient{125\,g Butter}
  \ingredient{125\,g Zucker}
  glatt rühren.
  
  \ingredient{125\,g Mehl}
  \ingredient{\sfrac{1}{2} Pkg. Backpulver}
  darauf sieben und unterrühren.
  
  \ingredient{125 ml Eierlikör}
  \ingredient{1 EL Schokoladeraspeln}
  unterrühren.

  Die Vertiefungen der Form zu \sfrac{2}{3} ihrer Höhe mit Teig füllen.
  
  Bei 200° (Ober-"/Unterhitze) ca. 15--20 Minuten im vorgeheizten
  Backofen backen.

  \ingredient{75\,g Zartbitterkuvertüre}
  \ingredient{25\,g Weisse Kuvertüre}
  Die Kuvertüren getrennt voneinander schmelzen. Muffins dunkel überziehen,
  etwas antrocknen lassen, mit weisser Kuvertüre garnieren (streifen).

  Ergibt 12 Stück.
\end{recipe}

\begin{recipe}{Kokos-Muffins}
  \ingredient{1 kl. Dose Ananasstücke}
  auf einem Sieb abtropfen lassen, den Saft auffangen.

  \ingredient{150\,g weiche Butter}
  \ingredient{100\,g Puderzucker}
  \ingredient{2 Eigelb}
  mit den Quirlen des Handrührers etwa 5 Minuten schaumig schlagen.

  \ingredient{160\,g Mehl Type 550}
  \ingredient{1 TL Backpulver}
  über die Schaummasse sieben und mit
  \ingredient{70\,g Kokosflocken}
  unter den Teig rühren.

  \ingredient{2 Eiweiß}
  \ingredient{50\,g Puderzucker}
  steif schlagen. Mit einem Schneebesen unter die Teigmasse ziehen.

  Ananas fein würfeln und unterheben.

  Backförmchen zu gut \sfrac{2}{3} mit Teig füllen.

  Bei 175° (Ober-"/Unterhitze) ca. 30 Minuten im vorgeheizten
  Backofen backen.

  Für die Glasur
  \ingredient{1\sfrac{1}{2} EL Ananassaft}
  \ingredient{75\,g Puderzucker}
  mit einem Schneebesen verrühren. Je einen Esslöffel über die lauwarmen
  Muffins geben und mit
  \ingredient{2 EL Kokosflocken}
  bestreuen.

  Ergibt 15 Stück.
\end{recipe}

\begin{recipe}{Kokosmuffins mit Kokossauce}
  \ingredient{150\,g Milch}
  \ingredient{150\,g Sahne}
  \ingredient{100\,g Kokosraspel}
  erhitzen und die Saucen-Mischung anschliessend mindestens 12 Stunden
  zugedeckt kühl ziehen lassen.

  \ingredient{100\,g Zartbitterkuvertüre}
  fein hacken. Mit
  \ingredient{320\,g Mehl}
  \ingredient{1 geh.\,TL Backpulver}
  \ingredient{150\,g Kokosraspel}
  mischen.

  \ingredient{250\,g Butter}
  \ingredient{200\,g Zucker}
  schaumig rühren und nach und nach
  \ingredient{5 Eier}
  und Mehlmischung zugeben.
  
  \ingredient{1 Banane}
  schälen, in kleine Würfel schneiden und unter den Teig heben.

  Den Teig in 12 mit Papiermanschetten ausgelegte Muffinformen füllen.

  Bei 180° (Ober-"/Unterhitze) ca. 20--25 Minuten im vorgeheizten
  Backofen backen. Dann aus der Form nehmen und auf einem Kuchengitter
  auskühlen lassen.
  
  \ingredient{100\,g Kokosraspel}
  auf einem Backblech oder in einer Pfanne leicht rösten.

  \ingredient{50\,g Zartbitterkuvertüre}
  im Wasserbad auflösen.

  Die Oberfläche der ausgekühlten Muffins mit der flüssigen Kuvertüre
  einstreichen und mit den gerösteten Kokosraspeln bestreuen.

  
  Die Saucen-Mischung durch ein feines Sieb abschütten und die
  verbleibende Flüssigkeit mit
  \ingredient{50 ml Batida de Coco}
  \ingredient{40\,g Zucker}
  erneut aufkochen. Etwas von der heissen Sauce mit
  \ingredient{4 Eigelb}
  mischen (angleichen), unter die restliche heisse Sauce rühren und zur
  Rose abziehen. Die Sauce auf Eiwasser kalt rühren und zu den Muffins
  servieren.
\end{recipe}

\begin{recipe}{Nutella-Muffins}
  \ingredient{80\,g weiche Butter}
  \ingredient{40\,g Zucker}
  \ingredient[Nutella]{100\,g Nutella}
  \ingredient{3 Eier}
  \ingredient{120\,ml Milch}
  in einer Rührschüssel mit dem Mixer schaumig rühren.
  
  \ingredient{300\,g Mehl}
  \ingredient{1 Pkg. Backpulver}
  \ingredient[Schokolade]{3 EL Schokoflocken}
  vermengen und vorsichtig unter die Eiermasse heben.

  Eine Muffinform mit Papierförmchen auslegen. Die Masse in die
  vorbereiteten Förmchen füllen.

  Bei 200° (Ober-"/Unterhitze) ca. 30 Minuten im vorgeheizten Backofen
  backen.

  Ergibt 15 Stück.
\end{recipe}

\begin{recipe}{Schokoladen-Muffins}
  \ingredient{125\,g weiche Butter}
  \ingredient{150\,g Zucker}
  \ingredient{1 Pkg. Vanillezucker}
  verrühren.

  \ingredient{2 Eier}
  unterrühren.

  \ingredient[Schokolade]{200\,g Zartbitterschokolade}
  grob hacken.

  \ingredient{200\,g Mehl}
  \ingredient[Kakao]{4 EL Backkakao}
  \ingredient{1 Prise Salz}
  \ingredient{2 TL Backpulver}
  vermischen.

  \ingredient{175\,ml Milch}
  mit der Mehlmischung zur Butter-Zuckermischung geben und alles gut
  verrühren. Etwa zwei Drittel der gehackten Schokolade unterheben.

  Die Mulden eines Muffinblechs mit Förmchen auslegen. Mit einem
  Eisportionierer den Teig auf die Förmchen verteilen. Die restlichen
  gehackten Schokostückchen auf den Muffins verteilen.

  Bei 160° (Umluft) bzw. 180° (Ober-"/Unterhitze) ca. 25 Minuten im
  vorgeheizten Backofen backen.

  Ergibt 15 Stück.
\end{recipe}

\begin{recipe}{Schokoladen-Muffins mit Kokosfüllung}
  \ingredient{200\,g Mehl}
  in eine Schüssel sieben und mit
  \ingredient{2 TL Backpulver}
  \ingredient[Kakao]{4 EL Kakaopulver}
  mischen.

  In einer zweiten Schüssel
  \ingredient{1 Ei}
  \ingredient{150\,g Brauner Zucker}
  \ingredient{100\,g weiche Butter}
  zu einer luftigen Masse aufschlagen.

  \ingredient{300\,ml Buttermilch}
  beifügen und einen Moment weiterschlagen. Zuletzt die Mehlmischung
  sorgfältig unterziehen.

  Für die Füllung
  \ingredient{100\,g Frischkäse}
  \ingredient{2 EL Zucker}
  \ingredient{1 Pkg. Vanillezucker}
  \ingredient{50\,g Kokosflocken}
  mischen.

  Die Muffins-Förmchen mit Öl bepinseln und mit wenig Mehl bestäuben
  (alternativ Papierförmchen verwenden). Gut ein Drittel des Teiges in
  die Förmchen geben. Jeweils ein Teelöffel Füllung daraufgeben und mit
  dem restlichen Teig decken.

  Bei 180° (Ober-"/Unterhitze) ca. 15 Minuten im vorgeheizten Backofen
  backen. Auf einem Kuchengitter aus"-kühlen lassen.

  \ingredient{1 Pkg. dunkle Kuchenglasur}
  nach Anleitung zubereiten. In eine kleine Schüssel geben, den oberen
  Teil der Muffins in die Glasur tauchen. Die Glasur sofort mit wenig
  \ingredient[Kokos]{Kokosflocken}
  bestreuen.

  Ergibt 30 Stück.
\end{recipe}

\begin{recipe}{Schokoladen-Walnuss-Muffins}
  \ingredient{175\,g Butter}
  \ingredient{150\,g Haushaltsschokolade, kleingehackt}
  im Wasserbadtopf schmelzen. In eine große Rührschüssel geben.

  \ingredient{200\,g Zucker}
  \ingredient{50\,g Brauner Zucker}
  unter die Schokoladenmasse rühren.

  Nach und nach
  \ingredient{4 Eier}
  untermischen, dann 
  \ingredient{1 TL Vanillearoma}
  \ingredient{\sfrac{1}{2} TL Mandelaroma}
  zugeben.
  
  \ingredient{110\,g Weizenmehl}
  \ingredient{1 EL Kakaopulver}
  darübersieben und unterziehen.
  
  \ingredient{115\,g Walnüsse; gehackt}
  unterheben.

  Die Muffinförmchen fast bis zum Rand füllen.

  Bei 180° (Ober-"/Unterhitze) ca. 30--35 Minuten im vorgeheizten
  Backofen backen, so dass ein Teststäbchen sauber, aber leicht klebrig
  bleibt. 5 Minuten stehen lassen und dann auf einem Kuchengitter
  erkalten lassen.

  Ergibt 12 Stück.
\end{recipe}

\begin{recipe}{Schokoladen-Whisky-Muffins}
  \ingredient{150\,g Blockschokolade}
  in Stückchen brechen, mit
  \ingredient{100\,g Butter oder Margerine}
  in einem kleinen Topf im Wasserbad bei schwacher Hitze zu einer
  geschmeidigen Masse verrühren.

  \ingredient{250\,g Weizenmehl}
  \ingredient{2 gestr.\,TL Backpulver}
  \ingredient{\sfrac{1}{2} gestr.\,TL Natron}
  in eine Rührschüssel sieben, mit
  \ingredient{150\,g Zucker}
  \ingredient{1 Pkg. Vanille-Zucker}
  \ingredient{1 Prise Salz}
  mischen.
  
  \ingredient{2 Eier (Größe M)}
  \ingredient{100 ml Starker Kaffee}
  \ingredient{4 EL Whisky}
  und die Schokoladen-Fett-Mischung
  hinzufügen. Mit dem Handrührgerät mit Knethaken zu einem glatten Teig
  verarbeiten.
  
  \ingredient{50\,g Raspelschokolade}
  unterrühren.
  
  Den Teig in 12 gefettete Muffin-Förmchen geben (oder Papierformen
  verwenden).

  Bei 180° (Ober-"/Unterhitze) ca. 25 Minuten im vorgeheizten Backofen
  backen. Die Muffins 10 Minuten in den Förmchen stehen lassen, dann aus
  den Förmchen lösen.

  Für die Creme
  \ingredient{1--2 Becher Crème Double (je 125\,g)}
  mit dem Handrührgerät mit Rührbesen anschlagen.

  \ingredient{2--3 EL Whisky}
  unterheben, mit
  \ingredient{1 Pkg. Vanille-Zucker}
  und
  \ingredient{Zucker}
  abschmecken.

  Die Creme auf die Muffins geben.

  Ergibt 12 Stück.
\end{recipe}

\begin{recipe}{Weihnachts-Muffins}
  \ingredient{100\,g Butter}
  \ingredient{80\,g Honig}
  cremig rühren.
  
  \ingredient{2 Eier}
  nacheinander unterrühren.
  
  \ingredient{100\,g Mehl}
  \ingredient{2 TL Backpulver}
  \ingredient{1 Prise Salz}
  \ingredient{100\,g Haferflocken, zart}
  \ingredient{80\,g Walnüsse, gemahlen}
  mischen und unter den Teig rühren.
  
  \ingredient{\sfrac{1}{8} l Milch}
  Je nach Konsistenz des Teigs eventuell nur einen Teil der Milch
  unterrühren.
  
  \ingredient{150\,g Rosinen}
  unterheben und die Masse mit
  \ingredient{1 TL Zimt}
  \ingredient{1 Prise Anis, gemahlen}
  würzen.

  Teig in Förmchen oder Muffinblech füllen.

  Bei 200° (Ober-"/Unterhitze) ca. 30--35 Minuten im vorgeheizten
  Backofen backen.

  Eventuell Muffins mit Kuvertüre überziehen und einer halben Walnuss
  verzieren.

  Ergibt 12 Stück.
\end{recipe}
